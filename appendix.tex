\appendix
\label{cod:matlab}
\newpage
\vspace*{-21.6pt}
\begin{center}
\zihao{3}\heiti
\phantomsection%设置“幻影标题”,解决额外插入目录(\addcontentsline)后总是引用到第一页的问题;
\addcontentsline{toc}{section}{附录}
附录
\end{center}

以下是布封投针求圆周率\,$\uppi$\,的Matlab的部分代码:
 %Matlab 代码格式设置
\definecolor{DarkGreen}{rgb}{0.0,0.4,0.0}
\lstloadlanguages{Matlab}
\lstset{language=Matlab,
	frame=shadowbox,                           % shadowbox framed
    rulesepcolor= \color{gray},%框的颜色
    breaklines=true,
	basicstyle=\small\ttfamily,
	keywordstyle=[1]\color{blue}\bfseries,  % primitive funs in bold blue
	keywordstyle=[2]\color{purple},         % args of funs in purple
	keywordstyle=[3]\color{blue}\underbar,  % user funs in blue with underbar
	stringstyle=\color{purple},             % strings in purple
	showstringspaces=false,
	identifierstyle=,
	commentstyle=\usefont{T1}{pcr}{m}{sl}\color{DarkGreen}\small,
	tabsize=4,
	% more standard MATLAB funcs
	morekeywords={sawtooth, square},
	% args of funcs
	morekeywords=[2]{on, off, interp},
	% user funcs
	morekeywords=[3]{FindESS, homework_example},
	morecomment=[l][\color{blue}]{...},     % line continuation (...) like blue comment
	numbers=left,
	numberstyle=\tiny\color{blue},
	firstnumber=1,
	stepnumber=1,
    escapeinside=``,
}

%插入 Matlab 脚本文件
\begin{lstlisting}[language=Matlab]
clear;clc;
n=2000;%`投针次数figure;`
axis([0,2,0,2]);
hold on;
plot([0,2],[0.5,0.5],'k','LineWidth',2);%`画y=0.5的直线`
plot([0,2],[1.5,1.5],'k','LineWidth',2);%`画y=1.5的直线`
x1=ones(1,n);y1=ones(1,n);%`投针中点坐标`
x2=ones(1,n);y2=ones(1,n);%`投针一端坐标`
x3=ones(1,n);y3=ones(1,n);%`投针另一端坐标`
jiaodu=ones(1,n);%`投针角度`
distant=ones(1,n);%`投针与最近直线的距离`
xianjiao=ones(1,n);%`投针与直线相交数`
for i=1:n %`开始投n次x1(1,i)=2*rand;`
    y1(1,i)=0.5+1*rand;
    jiaodu(1,i)=pi*rand;
    x2(1,i)=x1(1,i)-0.25*cos(jiaodu(1,i));
    y2(1,i)=y1(1,i)-0.25*sin(jiaodu(1,i));
    x3(1,i)=x1(1,i)+0.25*cos(jiaodu(1,i));
    y3(1,i)=y1(1,i)+0.25*sin(jiaodu(1,i));
    if(y1(1,i)<1)%`计算距离`
         distant(1,i)=y1(1,i)-0.5;
    else distant(1,i)=1.5-y1(1,i);
    end
end
n/sum(xianjiao)%`计算π`
\end{lstlisting}
